\documentclass{article}
\usepackage[utf8]{inputenc}
\usepackage{hyperref}

\usepackage{graphicx}
\usepackage{subcaption}

\graphicspath{ {./images/} }
\usepackage{wrapfig}

\usepackage{caption}
\captionsetup[figure]{font=footnotesize}

\usepackage[framemethod=TikZ, xcolor=RGB]{mdframed}
\definecolor{mycolor}{RGB}{225,225,225}
\newmdenv[
        linecolor=mycolor,
        topline=false,
        bottomline=false,
        rightline=false,
        linewidth=1pt,
        innerleftmargin=5pt,
        leftmargin=0pt,
        rightmargin=0pt,
        innerbottommargin=0pt
]{separator}

\usepackage[backend=biber,
bibencoding=ascii,
style=ieee,
citestyle=numeric,
sorting=ynt
]{biblatex}
\addbibresource{bibliography.bib}

\title{Collaborative Learning: \\
\normalsize Strategies and Platform Development
}
\author{Paul Baier}
\date{April 20, 2020}

\begin{document}

\maketitle

\section{Introduction}
A student response system (SRS) is a tool used in classrooms to get student feedback in real time. Typically it would be an online tool that teachers can set up with questions and present to students, and students can connect and answer the questions providing real time results. Gamification is the process of applying learning outcomes to games. The goal of this project is to create a fully customizable and programmable gamified SRS platform. We categorize such a system as a ``student response game'' (SRG). 

    \subsection{Kahoot}
        Our system is based off an existing commercial platform called ``Kahoot!''. Kahoot is a platform that allows teachers to write questions and organize them together to create a game called a ``kahoot''. When played, the teacher projects the questions to the students, as seen in figure \ref{fig:kahoot-question} and the students answer the questions on their own devices. The quicker a student answers the question, the more points they get. No points are awarded or subtracted for incorrect answers. After each question the results are displayed for that question and then the scoreboard with each player's cumulative score is displayed, as in figures \ref{fig:kahoot-post_question} and \ref{fig:kahoot-scoreboard} respectively. When the game is finished an animation plays and the final results are displayed as shown in figure \ref{fig:kahoot-final}.
        
        \begin{figure}[ht]
            \centering
            \begin{subfigure}[b]{0.49\textwidth}
                \includegraphics[width=\textwidth]{images/kahoot-question.png}
                \caption{Multiple Choice Question}
                \label{fig:kahoot-question}
            \end{subfigure}
            \begin{subfigure}[b]{0.49\textwidth}
                \includegraphics[width=\textwidth]{images/kahoot-post_question.png}
                \caption{Question Results}
                \label{fig:kahoot-post_question}
            \end{subfigure}
            \\
            \begin{subfigure}[b]{0.49\textwidth}
                \includegraphics[width=\textwidth]{images/kahoot-scoreboard.png}
                \caption{Scoreboard}
                \label{fig:kahoot-scoreboard}
            \end{subfigure}
            \begin{subfigure}[b]{0.49\textwidth}
                \includegraphics[width=\textwidth]{images/kahoot-finished.png}
                \caption{Final Results}
                \label{fig:kahoot-final}
            \end{subfigure}
            \caption{Kahoot Screens \cite{kahoot}}\label{fig:kahoot}
        \end{figure}
        
        When creating questions teachers are given six choices for the type of question they want to create: multiple choice, true/false, open-ended (players write in their answer), puzzle (players need to order the answers correctly), poll (no points awarded, simply polling the players), and slide, which is not a question but rather a screen with text to display more information. Each question has text and teachers are given the option to include a picture or YouTube video.
        \smallskip
        
        A Kahoot can be played live in the classroom or assigned to students to complete at their own pace. If assigned, the teacher needs to enter a due date and can set a few game options including randomize the questions, randomize the answers in each question, and use the question timer. Live games can be played in either classic mode, where students compete individually, or team mode. The difference between the two modes is that students share a device in team mode. Game options include randomizing the question and answers (like in the assigned option) and automatically moving through the questions instead of the teacher needing to click the ``next'' button as seen in figure \ref{fig:kahoot-scoreboard}.
    
    \subsection{Motivation}
        We decided to create a flexible, programmable SGR because we like the idea of Kahoot but find it limiting in the game options it offers. The idea originated with student collaboration in mind - creating a platform like kahoot but more collaborative - because Kahoot's team mode leaves a lot to be desired. As we iterated over different designs and features to include we eventually arrived at the idea that users (teachers) should be able to program the game and its features to do whatever they want it to do. With this concept at the core of our platform, we designed an open system with all of the game logic and data exposed to the user. There are default behaviors built into the system for those that want a hassle free experience, but for the more ambitious and/or particular, the default behavior can be overridden and programmed in any number of different ways.  

\section{Literature Review}
In order to identify prior work in the area of educational games we looked at roughly 2000 proceedings from the Frontiers in Education (FIE) conference from 2015-2019. We started our search with the word ``kahoot'' because our platform's functional design is based off of Kahoot. Within the search results we identified other platforms mentioned that were similar to Kahoot (and therefore also similar to our platform) and recursively searched for those other platforms as well within the proceedings. This search method allowed us to determine other popular platforms researchers were testing that are comparable to Kahoot as well as other SRS/SRG platforms in development. Our search revealed some of the most popular and relevant SRGs currently in production in addition to Kahoot are Socrative \cite{socrative}, Quizizz \cite{quizizz}, and Quizlet \cite{quizlet}. There were also two platforms in development by researchers called Quipid \cite{quipid} and Dysgu \cite{dysgu}.

    \subsection{Quizizz}
        Of the existing platforms Quizizz most resembles Kahoot and has many customizable features. The main difference between it and Kahoot or our platform is that students move through the quiz at their own pace, limited by the time assigned to each question.
        \smallskip
        
        To get started a teacher can either choose a preexisting quiz from a library of quizzes organized by topic or they can create their own. If they choose a pre-made quiz, they can use it as is or copy and edit it to their liking. If they choose to create a new quiz, they can search through existing quizzes for questions in addition to creating their own.
        \smallskip
        
        Question types include multiple choice, checkbox (select all the correct answers), fill-in-the-blank, polls, and open-ended (the last two are ungraded but marked as correct in reports). For each question the teacher can specify the amount of time the student has to answer. Scores for answering a question correctly are determined in part by the time it took the student to answer relative to the time length of the question. However, the timer can be disabled in which case the time factor would not be considered when scoring a question.
        \smallskip
        
        Once the quiz is made there are three game modes: team, classic, and test. We will only look at team and classic modes as test mode is beyond the scope of our work.
        \smallskip
        
        Team mode, as the name suggests, organizes students into teams. In classic mode each student competes by themselves. The main difference between the two modes is that the number of teams needs to be specified by the teacher in team mode and the team scores are based on the cumulative results of the team members. Both team and classic mode share similar gameplay settings. A teacher is able to toggle the timer on or off for the game, shuffle the questions, shuffle the answer options, and allow for a ``redemption question'' where students are able to retry missed questions. There are also memes that show up between questions that the teacher can toggle off for the game or the student can toggle off for their own experience. Another option teachers have is to either show the correct answer to the student in the game after they answer, show only if the got the question right or wrong, or do not visually indicate a correct or incorrect answer (there is a noise that indicates correct or incorrect regardless of this setting).
        \smallskip
        
        Lastly Quizizz has a feature called ``Power-ups'' which ``are single-use abilities designed to increase engagement and participation'' \cite{quizizz}. Currently there are nine power-ups that are awarded sporadically to players. Once awarded, an icon appears on the student's screen and they can use it any time by clicking on the icon. An example of a power up is 50-50 which eliminates half of the incorrect answers. This entire system is available in both team and classic mode and can be disable before starting the game.
        \smallskip
        
        Described above are only some of the features offered on the Quizizz platform but they represent the most relevant when compared with our design.
    
    \subsection{Socrative}
        Socrative is a web application focused on student engagement and tracking. It offers a number of features outside of the scope of our application, however one of its activities, called ``Space Race'', is comparable.
        \smallskip
        
        Space Race is a quiz-like game where students are asked a series of questions prearranged by the teacher. Figure \ref{fig:socrative-space-race} shows how student progress is tracked by little rocket ships (or other icons) that move across the teacher's dashboard when a question is answered correctly. The goal is to be the first individual or team to have their rocket reach across the screen.
        \smallskip
        
        \begin{figure}[ht]
            \centering
            \includegraphics[width=0.8\textwidth]{images/socrative-space_race.jpg}
            \caption{Socrative Space Race \cite{socrative}}
            \label{fig:socrative-space-race}
        \end{figure}
        
        Quizzes can be made of multiple choice, true/false, and short answer questions. Other quiz options include the number of teams, how the players are assigned to teams (auto-assigned or student's choice), if the quiz is timed, shuffle the questions, shuffle the answers, show question feedback, show the final score, and if questions can only be attempted once.
    
    \subsection{Quizlet}
        Like Socrative, Quizlet is a web application that offers many features to its users. It provides students and teachers with the tools to create interactive study materials based on ``sets'', which are essentially digital flashcards. Each set is made up of any number of ``cards'' which contain a term ``side'' and a definition ``side''. Teachers and students can input the term and definition for their cards and create their own sets.
        \smallskip
        
        While most of the features offered by Quizlet do not overlap with our platform there is one, called ``Quizlet Live'', that is similar.
        \smallskip
        
        Quizlet live is a game that asks students multiple choice questions. Figure \ref{fig:quizlet-live} is the view from the teacher's dashboard that tracks student progress in real time. The teacher selects the set they want to use for the game and is able to choose between individual and team mode. Once selected the teacher is given the option for the questions to be generated based on the term side of the card or the definition side. The game ends for everyone when the first individual or team answers all of the questions correctly. If a question is answered incorrectly the correct answer is shown and the individual or team starts back at the beginning. Questions are randomly displayed to each player/team. In individual mode there are always four answer options shown in a random order, three of which are answers for other questions in the set.
        
        \begin{figure}[ht]
            \centering
            \includegraphics[width=0.6\textwidth]{images/quizlet-live.png}
            \caption{Quizlet Live Dashboard \cite{quizlet}}
            \label{fig:quizlet-live}
        \end{figure}
        
        In team mode the number of teams is automatically generated based on the number of players. Students are assigned to a team randomly and the teams can be shuffled by the teacher before the game starts. In the game, the answers to each question are split equally between each member of the team and always displayed on the screen. There are no wrong answers displayed in team mode.
        \begin{wrapfigure}{l}{0.25\textwidth}
            \centering
            \includegraphics[width=0.15\textwidth]{images/quizlet-team.png}
            \caption{Quizlet Team View \cite{quizlet}}
            \label{fig:quizlet-team}
        \end{wrapfigure}
        Figure \ref{fig:quizlet-team} shows an example of one player's screen with two answer options, one of which has already been submitted correctly and replaced with a check mark. The other answer option, ``4'', will need to be used in an upcoming question.
        \smallskip
        
        To illustrate the point further, if there were 10 questions and two people on one team, then each person on that team would see five answers on their screen. Each answer is the answer to one of the questions but because the answers are divided across all team members, a player will not always have the answer to a question on their screen. It is up to the player with the correct answer to answer the question. Once the question is answered correctly a check mark appears where that answer used to be and the next question is displayed.
    
    \subsection{Quipid}
        A tool that was presented at the FIE 2017 conference is called Quipid. Its purpose is to provide students with rapid feedback on assessments and teachers with a simple way to create and assign assessments. It is not a game, so it does not directly relate to our platform, however it allows teachers to create custom questions and has a programmable element which makes it related and unique relative to other platforms we looked at.
        \smallskip
        
        Quipid uses Google Sheets and Google Forms as a front end for the platform. The teacher uses Google Sheets to formulate questions and organize them into a quiz. The quiz is exported to a Google Form where the students go to take the quiz. Once complete, the students are able to see their score along with feedback. Teachers can also see real time results once a student submits their work.
        \smallskip
        \begin{wrapfigure}{r}{0.25\textwidth}
            \centering
            \includegraphics[width=0.25\textwidth]{images/quipid-problem.png}
            \caption{Quipid Problem Structure \cite{quipid}}
            \label{fig:quipid-problem}
        \end{wrapfigure}
        \indent A main goal for Quipid is to provide teachers with an easy way to assess students. Google Sheets is the backbone to achieving this goal. Quipid is designed with two sheets, a ``Problem'' sheet and a ``Formulation'' sheet. The Problem sheet is a library of questions and is linked to the formulation sheet. As figure \ref{fig:quipid-problem} shows, a problem is made up of three parts: a ``question steam'', an optional ``figure'', and ``alternatives'', which are the multiple choice options for the question. The link between the formulation sheet and the problem sheet allows the formulation sheet to be programmed in such a way that when a teacher changes the parameters in the question stem, the alternatives are automatically updated.
        \smallskip
        
        Along with the formulation sheet, Quipid offers more programmability simply by virtue of using Google's G Suite (Google Sheets and Google Forms). Part of G Suite is a scripting language called Google Apps Script. This allows the user to program the Sheets and Forms with virtually infinite possibilities. Quipid leverages Apps Script to create the quiz in Google Forms based on the questions in Google Sheets, but a user of Quipid could program the tool to do any number of different tasks including integrating with third party tools, at which point the possibilities would be endless.  
        
    \subsection{Dysgu}
        The last work, Dysgu, is a concept more than an application. It was presented at FIE 2018 in a ``work in progress'' paper. It is presented as a collection of high level ideas without any concrete implementation. We mention it here because it is a platform in development with gamification features and some interesting ideas on student engagement.
        \smallskip
        
        Dysgu is described as an application for teachers to assign activities to students and track their progress. Students can sign in, interact with each other, and complete the assignments. There are three notable gamification features presented: scores and points, badges, and social awareness.
        \smallskip
        
        The paper describes the difference between scores and points as ``Scores are utilized to calculate student’s grade, whereas, points are used as currency in the system'' \cite{dysgu}. All students receive scores for completing activities, but they can earn points by achieving certain goals within the activities, such as being the first to complete an activity. The points earned can then be used in various ways throughout the system, like a time extension on an activity. 
        \smallskip
        
        Badges are similar to points. They are awarded for certain accomplishments and unlock features for the student like getting a sneak peek into an activity.   
        \smallskip
        
        The social aspect of the platform allows students to see how they are doing relative to the rest of the class. While everything is done anonymously, certain general statistics are made available to each student. For example each student can see the number of students to have completed an activity and each student will know which place they are in relative to their points. Also each student will be able to see the others' badges. 

\section{Design}
Our platform is divided into a backend and a frontend. The backend handles the basic web app functionality along with all of the game logic and data storage. The frontend is responsible for dispaying data provided by the backend and processing user interaction. There is a very loose coupling between the backend and frontend. One advantage of having a loosely coupled back and front end is that they can be developed independently once an application programming interface (API) is agreed upon, allowing for quicker, parallel development. Another advantage is that either can be swapped out for a different technology should the need or desire arise as long as the new technology follows the same API.
\smallskip

An alternative approach would have been to write the frontend using the web framework's view implementation. The advantage with that approach is that the frontend would have greater visibility into and knowledge of the backend's data. This greater visibility comes at the cost of a potentially bloated system that could become difficult to change over time.
\smallskip

The overall platform design follows the microservice/model-view-template (MVT) pattern. Both patterns are well known and therefore will not be the focus of this paper (although we will briefly discuss certain architectural elements). Rather the design elements discussed here will be those specific to the actual game and not the surrounding architecture. In our implementation the backend is written in Python Django and the frontend is React.

	\subsection{Game Architecture}\label{architecture}
	    A game, at its core, is made of states, each representing a moment in the game with a specific functionality. We refer to this as Flow and Functionality. Flow is the progression from state to state and functionality is the operations that happen within each state. Our goal is to provide a flexible, programmable platform, so our focus throughout development has been to abstract out the flow and functionality of a game to provide as much control as possible to the user.

        \subsubsection{Flow}\label{flow}
	        Within a game our platform has seven states that can be arranged by the user in any order. The states are ``registration'', ``make\_teams'', ``standby'', ``questions'', ``leaderboard'', ``finished'', and ``hook''. Each state has logic in the backend that produces data to be displayed on the frontend. The flow of the game can be specified by the user and stored in the database as part of the game properties, discussed further in the \nameref{database} section \ref{database}. Figure \ref{fig:architecutre-flow} shows an example of how states might be arranged in a particular game with two questions. In an alternate configuration, if the user wanted to use the same game but run through the questions quicker, they could remove the leaderboard state between the questions while everything else about the game remains the same, as seen in figure \ref{fig:architecutre-flow2}.

            \begin{figure}[ht]
                \centering
                \begin{subfigure}[b]{0.45\textwidth}
                    \includegraphics[width=0.8\textwidth]{images/architecture-flow.png}
                    \caption{Sample Game Flow Outline}
                    \label{fig:architecutre-flow}
                \end{subfigure}
                \begin{subfigure}[b]{0.45\textwidth}
                    \includegraphics[width=0.8\textwidth]{images/architecture-flow2.png}
                    \caption{Sample Game Flow Outline Without Leaderboard State}
                    \label{fig:architecutre-flow2}
                \end{subfigure}
                \caption{Sample Game Flows}\label{fig:architecutre-game_flows}
            \end{figure}

            The ``hook'' state, called a flow hook, acts as a blank state where any behavior can be programmed to run when that state runs. Currently the flow hook backend state maps to the flow hook frontend state by specifying certain key values in the backend output in order to display it on the frontend. For example, specifying a value for the ``h1'' key in the payload from the backend will display that key's value in an h1 tag on the frontend. The current key options are h1, h3, h5, and p. We are hoping in the future to make the entire hook frontend page customizable.
        
        \subsubsection{Functionality}\label{functionality}
            Functionality is the algorithms that run during each state. To give users access to the underlying functionality of a game we have introduced the idea of hooks. A hook is a piece of user defined python code that will run when the user specifies it to run within the game flow. It is important to make a distinction between functionality hooks and flow hooks mentioned above. A flow hook is to be used when the user wants a custom view to be displayed during the game. A functionality hook is not a state itself, but rather runs in addition to the current
            \begin{wrapfigure}{r}{0.40\textwidth}
                \centering
                \includegraphics[width=0.40\textwidth]{images/architecture-hook.png}
                \caption{Adding a Functionality Hook to a Game Flow}
                \label{fig:functionality-hook}
            \end{wrapfigure}
            state.
            
            Figure \ref{fig:functionality-hook} shows how a functionality hook can be added to a game flow to run before the leaderboard state in order to double every player's points for the previous question. The user would need to define the algorithm for how they want to double all of the scores, but the example illustrates that this type of hook only alters the game's underlying data and does not require it's own state with a frontend view like a flow hook. Functionality hooks can be run before or after any state. Each state should only need to run at most one hook before it an one hook after it, however it would be possible to insert empty flow hooks to run more hooks before or after a particular state.

	\subsection{Backend}
        \subsubsection{Architecture}
    	    The server side code is written in Python using the Django web framework. We chose Django for our base framework for a number of reasons. First, Python is a familiar language that is good for rapid development because there is not a lot of boilerplate code required. Second, Python is a popular and well documented language and Django is a popular framework so any issues or gaps in our knowledge would likely have easily accessible solutions in either online forums or the Django documentation. Third, because of Django's popularity there are many external libraries that help extended its functionality. Two such libraries that we relied upon are Django REST framework (DRF) and Django Channels.
    	    \smallskip

            During a game, communication between the client and server uses the websockets protocol. In the OSI model websockets is a layer 7 protocol and an alternative to HTTP. The reason for using this protocol is that it maintains a client/server connection, giving the server the ability to send data to the client without the client requesting it. This allows for two-way, real time communication between the client and the server and is a crucial part of a real time, interactive game like ours. Because websockets is a standard protocol, the implementations on both the front and back ends are straightforward. On the frontend no extra tooling is required - websockets support is build into React. On the backend, Django Channels abstracts out many of the finer details of websockets, providing an easy library to  utilize the protocol.
        
        \subsubsection{The Game Object}\label{game-object}
            The brain of a game is the Game object. It is responsible for all game logic and communication. When first initialized it fetches all of the data from the database needed to run the game. This includes all of the questions, answers, game settings, and any custom code like functionality hooks. The game object extracts the game flow from the database and organizes the game accordingly. It runs through the states, from registering users to asking questions to scoring answers, and organizes the data that will be sent to the host and to the players for each state.
            \smallskip
            
            A major component of our platform is programmability through hooks. In addition to the hooks described in sections \ref{flow} and \ref{functionality} we also provide scoring hooks so the user can customize how the game is scored for individual and team games. For hooks to be effective they need to have access to the underlying data in the game object. Therefore we pass references of the game object into every hook so that users are able to read and manipulate any of the game data. This provides users with limitless access to the backend, as if they were themselves developers.
            \smallskip
            
            The flexibility and customization that hooks provide comes at the cost of security. To open the system up to users also opens it up to bad actors. Currently we do not account for security concerns but it is worth noting that handling hook security is both critical and nontrivial. In order to make it more secure there is likely to be some amount of functionality loss, and although the extend is unknown, it seems unlikely to be significant. Nevertheless the security concern must be addressed.
	    
		\subsubsection{Libraries}
		    Part of our application's backend architecture relies on external libraries outside of the Django framework core. Three of these libraries play major roles in the design of the system and are worth explaining here.
            \\\\
		    \noindent{\emph{Django REST Framework}}
		    \smallskip

    	    Django REST framework extends Django to act as a RESTful API. This allows us to program RESTful endpoints on the server side so our frontend can call the server using standard web API calls to fetch and deliver information. Part of this process requires authenticating users and authorizing user access to data and resources. For this we elected to use JSON web tokens (JWTs).
            \\\\
    	    \noindent{\emph{Simple JWT}}
    	    \smallskip

    	    A DRF library that we use that simplifies the JWT implementation process called Simple JWT \cite{simplejwt}. There are five reasons why we chose JWTs. The first is that they are a modern and cutting edge authentication process, and when it comes to security it is important to be at the forefront of technology. The second is that it provides both authentication and authorization in one package, simplifying what can be a bulky and complicated process. The third is that they are fast, secure, and compact to generate and send between the client and server. The fourth is that a JWT is easily stored and renewed on the frontend and easily expired or revoked on the backend.
            \\\\
    	    \noindent{\emph{Django Channels}}
    	    \smallskip
	    
	        The backbone of the game functionality is Django Channels. It is the library we use as the websockets implementation on the server side. Channels simplifies communication between the server and clients, and allows us to specifically track, target, and distinguish communication between the game host and the game players. Figure \ref{fig:backend-libraries_channels} shows the basic Channels structure. The Game object (shown in blue) handles the game logic as described in section \ref{game-object}.  When initializing a game the host connects to the backend via websockets. Any information the host needs to display comes directly from that websocket connection.

            \begin{figure}[ht]
                \centering
                \includegraphics[width=0.8\textwidth]{images/backend-libraries_channels.png}
                \caption{Django Channels Game Structure}
                \label{fig:backend-libraries_channels}
            \end{figure}
            
	        Once the game is created, players can join through their own websocket connection. Each player has a separate connection to the server and their connection instance knows which game to communicate with based on the game pin they used to join. Upon first connecting to the game, each instance registers with the Game object, so the game has knowledge of and can reference each player instance individually, allowing it to send unique data to specific players. All data sent to the players starts in the Game object, then is sent to each player's websocket instance, and finally is forwarded to each player's frontend view. Similarly, data sent from each player to the game, for example when players answer a question, starts with user input on the frontend, is sent through the websocket connection to that player's websocket instance, and then forwarded to the Game object. There the information is processed accordingly and because each websocket instance is registered with the game, the game can properly attribute the data to a specific player.
	        \smallskip
	        
		\subsubsection{Database}\label{database}
		Part of the backend's responsibilities is storing all of the system's data. This includes user names and passwords, games, game settings, questions, and answers. The database was designed with the goal of reusing as many objects as possible and follows third normal form as there are no funcitonal dependencies between table attributes. Figure \ref{fig:database-er_diagram} shows the Entity Relationship (ER) diagram of our database design.
		
        \begin{figure}[ht]
            \centering
            \includegraphics[width=0.9\textwidth]{images/database-er_diagram.png}
            \caption{Database Entity Relationship Diagram}
            \label{fig:database-er_diagram}
        \end{figure}
        
        It is helpful to think of the database as split into two sides, the question side, as in figure \ref{fig:question-tables}, and the game side, as in figure \ref{fig:game-tables}. The idea with this approach is that questions can be assigned to multiple games without the need to duplicate the question in the database.
        
        \begin{figure}[ht]
            \centering
            \begin{subfigure}[b]{0.45\textwidth}
                \includegraphics[width=\textwidth]{images/database-questions.png}
                \caption{Question Tables}
                \label{fig:question-tables}
            \end{subfigure}
            \begin{subfigure}[b]{0.45\textwidth}
                \begin{separator}
                    \includegraphics[width=\textwidth]{images/database-game.png}
                    \caption{Game Tables}
                    \label{fig:game-tables}
                \end{separator}
            \end{subfigure}
            \\
            \begin{subfigure}[b]{0.9\textwidth}
                \includegraphics[width=\textwidth]{images/database-game_questions.png}
                \caption{Question and Game Tables Relationship}
                \label{fig:question-game-tables}
            \end{subfigure}
            \caption{Database Sections}\label{fig:database-sections}
        \end{figure}
		
		A question is made up of text, a category, and whether it can be publicly shared or private. A question's answer options are stored in a separate table with a foreign key (FK) relationship to the question. This design allows a question to have, in theory, an infinite amount of answer options. In the prior work referenced, answer options were typically limited to four or less, but our design removes that restriction. Answer options are assigned to a specific question and have a bit value indicating if they represent the answer to the question. This allows single questions to have multiple answer options.
		\smallskip
		
		A flaw in this design is that the same answer option cannot be shared between questions because each option has a question id FK field.  To normalize this further we could add a junction table between the question and answer option so that the same answer options would not be duplicated in the database, however this approach seemed impractical because we did not want users having to search through existing answer options in order to add them to a question.
		\smallskip
		
		The main game table contains game properties and customization features. The two scoring hook fields (individual and team) FK to the scoring hook table as shown in figure \ref{fig:game-tables}. This allows for custom scoring algorithms to be written by our users and, if specified, will override the default scoring of our system as mentioned in section \ref{game-object}. The reason scoring hooks are in a separate table is so that the user can reuse the same code in multiple games. For example, if a user writes an algorithm in a scoring hook to assign points based on the order the answers were submitted, they could write that once to the scoring hook table and then assign that same hook to multiple games.
		\smallskip
		
		The outline column in the game table is where the game flow is stored. The functionality of the game flow is discussed in section \ref{flow}. There are many ways the game flow could be stored in the database. One option would have been to create a separate table with an FK to the game table, an ordinal column indicating the order of the states, and a state column indicating the specific state. It would also need some way, either with more columns or another table, in order to reference functionality hooks. This approach was considered but ultimately declined because of the complexity. The approach chosen seemed reasonable and much more concise, although we loose data integrity validation in exchange for the simplicity.
		\smallskip
		
		Our approach simply stores the game flow as a JSON string. While it is easier to corrupt this data, it provides much less overhead when storing and retrieving it and more importantly, allows for a more flexible shape of the data. Rather than relying on the database to ensure the data's integrity, we offload that responsibility to the front and back ends. When a game is fetched from the database, the game flow is retrieved from the outline column as a string and deserialized into its own data type.
		\smallskip
		
		This approach works well because it does not require any joins when retrieving the data and it allows for the shape of the data to be fluid. For example, each state in the flow may or may not have a functional hook before it and after it. By storing the flow as a string, if these values are not present, we can ignore them completely by not including them in the JSON string. If this were stored relationally we would have needed nullable pre-hook and post-hook columns for each state, adding a lot of potentially wasted space.
		\smallskip
		
		There are three tables in the database that are related to the game table which are the active game, the scoring hook, and the hook tables. Active games are games that are currently being played. They have an auto generated slug field that acts as the game pin so users can connect to it. The scoring hook table FKs to the game table and is used to store the code for both custom individual and team scoring.
		\smallskip
		
		The last table related to the game table is the hook table. Interestingly the hook table has no FKs to any other table (with the exception of the user table). This is because hooks do not belong to any one game and a game will have an unknown number of hooks. A junction table could be possible to connect hooks to games, but because of our outline design we elected for a different approach. In the hook table the creator and name fields must be unique together. That means that a single user cannot create two hooks with the same name. Because functional hooks are specified in the outline, it is there that the hook is referenced, namespaced using dot notation as ``creatorId.name''. For example if user 7 defined and wanted to use a functional hook called ``doublescore'' they would include it in the outline JSON as ``7.doublescore''. This way when the outline is deserialized, the hooks are parsed and the hook table is queried by creator id and name in order to fetch the hook's code.
		
		The question and game tables are joined using the question-game junction table, illustrated in figure \ref{fig:question-game-tables}. This table represents questions being assigned to games. It has time limit field as well so that question times are only specific to games and not questions. This allows the same question object to be used more flexibly. If a question's time limit was specified in the question table, then a duplicate tuple would need to be created just to change the time.

	\subsection{Frontend}

\section{Development}
\section{Features}
\section{Future Work}

\printbibliography[title={Bibliography}]


\end{document}

